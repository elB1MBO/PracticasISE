\documentclass[a4paper]{article}
\usepackage[spanish]{babel}
\usepackage[pdftex,usenames,dvipsnames]{color}
\usepackage{multicol}
\begin{document}
  \pagestyle{plain}
  \begin{center}
    \Huge\textsc{\textbf{\underline{TITULO}}}
  \end{center}
  \label{texto_rojo}\normalsize\textcolor{Red}{ESTE TEXTO ESTÁ EN ROJO}
  \clearpage
  Segunda pagina: \newline
  \ref{texto_rojo}Referencia al texto en rojo. Enumerado:\newline
  \begin{enumerate}
    \item Primer ítem,
    \item Segundo ítem, y
    \item Tercer \'{\i}tem.
  \end{enumerate}
  Descripción:
  \begin{description}
    \item[Nombre] Raúl
    \item[Apellidos] Durán Racero
    \item[Apodo] Bimbo   
  \end{description}

  Podemos anidar las listas:
  \begin{itemize}
    \item Frases fesquitas del juja:
    \begin{enumerate}
      \item Hola bom día
      \item Finalmentch
      \item Gente soy regil
    \end{enumerate}
    \item pleitos pleitos
    \begin{enumerate}
      \item Primero
      \begin{enumerate}
        \item Segundo
        \begin{enumerate}
          \item Tercero
          \begin{enumerate}
            \item Se pueden anidar como máximo 4
          \end{enumerate}
        \end{enumerate}
      \end{enumerate}
    \end{enumerate}
  \end{itemize}

  Tablita:
\begin{tabular}{|p{3cm}|p{5cm}|c|}
  \hline 
  Edad & Altura & Peso  \\
  \hline 
  5 & 105 & 25  \\
  \hline 
  10  & 120 & 35  \\
  \hline
\end{tabular}

Este es un texto \footnote{nota al pie de pagina de este texto} 
con una nota al pie de página, y aquí hay otra\footnote{matoncillo puto} 
nota.
\label{Latex_Basico}Introducción a LaTeX
\newpage
\begin{thebibliography}{99}
  \bibitem[Opc1]{Etiqueta1} primer elemento de la bibliografía
  \bibitem[Opc1]{Etiqueta2} segundo elemento de la bibliografía
  \bibitem[Opc2]{Etiqueta3} Para aprender a crear este archivo \LaTeX
  se ha utilizado el guión de Introducción a \LaTeX de Jeronimo Alaminos. 
  \cite[2]{Jeronimo Alaminos}
\end{thebibliography}

\end{document}