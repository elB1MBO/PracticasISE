\documentclass[a4paper]{article}
\usepackage[spanish]{babel}
\usepackage[pdftex,usenames,dvipsnames]{color}
\usepackage{multicol}
\usepackage{graphicx}
\usepackage{listings}
\usepackage{color}

\definecolor{dkgreen}{rgb}{0,0.6,0}
\definecolor{gray}{rgb}{0.5,0.5,0.5}
\definecolor{mauve}{rgb}{0.58,0,0.82}

\lstset{
    language=bash, 
    basicstyle=\footnotesize\color{white},
    backgroundcolor=\color{black},
    morekeywords={durar}, keywordstyle=\color{green},
    classoffset=1,
    showspaces=false,
    showstringspaces=false,
    showtabs=false,
    frame=single, 
    tabsize=2,
    captionpos=b,
    breaklines=true,
}

\begin{document}
\pagestyle{plain}
\title{Práctica 3: Monitorización y "Profiling" \\ 
Ingeniería de Servidores}
\author{Raúl Durán Racero}
\begin{figure}
    \centering
    \includegraphics[width=1.25\textwidth]{servers.pdf}
\end{figure}
\maketitle
\begin{figure}
    \centering
    \includegraphics[width=0.25\textwidth]{logoEtsiit.pdf}
\end{figure}

\newpage
\tableofcontents
\newpage
\section{Ejercicio 1}
Realice una instalación de Zabbix 5.0 en su servidor con \textbf{Ubuntu Server20.04} y configure
para que se monitorice a él mismo y para que monitorice a la máquina con \textbf{CentOS}.
Puede configurar varios parámetros para monitorizar, uso de CPU, memoria, etc. pero
debe configurar de manera obligatoria la monitorización de los servicios \textbf{SSH} y \textbf{HTTP}.
\subsection{Instalación de Zabbix}
Por comodidad a la hora de instalar, me conectaré a la máquina virtual de UbuntuServer
a través de SSH, para poder copiar algunos comandos que son muy largos.\newline
Lo primero que tenemos que hacer es instalar Zabbix en Ubuntu. Para ello, 
descargamos su paquete:

\begin{lstlisting}
durar@durar:~$ wget https://repo.zabbix.com/zabbix/5.0/ubuntu/pool/main/z/zabbix-release/zabbix-release_5.0-1+focal_all.deb
\end{lstlisting}
\end{document}